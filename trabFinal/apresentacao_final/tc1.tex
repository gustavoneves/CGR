\documentclass{beamer}

\usepackage[brazil]{babel}
\usepackage[utf8]{inputenc} 

\usepackage{amsmath}
\usepackage{multirow}

    \makeatletter
    \def\verbatim{\tiny\@verbatim \frenchspacing\@vobeyspaces \@xverbatim}
    \makeatother
       
    \pgfdeclareimage[height=1.2cm]{logo}{./imagens/logoUdescFinal}
    \logo{\pgfuseimage{logo}}
    
    \pgfdeclareimage[height=6cm]{duvidas}{./imagens/ponto1}

    \pgfdeclareimage[height=5cm]{resultado}{./imagens/resultado}

    \usetheme{Berkeley}

  
%%% Numerar slides
%%% Ref: https://tex.stackexchange.com/a/145899
%%% https://tex.stackexchange.com/questions/145829/slide-number-in-beamer-berkeley-header
\makeatletter
\setbeamertemplate{frametitle}{%
    \nointerlineskip%
    \vskip-\beamer@headheight%
    \vbox to \beamer@headheight{%
      \vfil
      \leftskip=-\beamer@leftmargin%
      \advance\leftskip by0.3cm%
      \rightskip=-\beamer@rightmargin%
      \advance\rightskip by0.3cm plus1fil%
      {\usebeamercolor[fg]{frametitle}
          \usebeamerfont{frametitle}\insertframetitle\hfill\insertframenumber\par}% added number
      {\usebeamercolor[fg]{framesubtitle}
           \usebeamerfont{framesubtitle}\insertframesubtitle\par}%
      \vbox{}%
      \vskip-1em%
      \vfil
    }%
  }
\makeatother

    
    \title{Sweeping}
    \author{Gustavo José Neves da Silva}

    \date{}
    
    
    \begin{document}
    
    \begin{frame}
      \titlepage
    \end{frame}
    
\section{Base}

\begin{frame}
  \frametitle{Teorema de Weierstrass}
    "Toda função contínua pode ser abitrariamente aproximada por um polinômio"
\end{frame}

\begin{frame}
  \frametitle{Interpolação}
  Polinômio de grau no máximo $n$ que coincide com $f(x)$ em $x_0, x_1, ..., x_n$.\\
  Tal polinômio de designado por $P_n(f; x)$, para maior clareza $P_n(x)$
\end{frame}

\begin{frame}
  \frametitle{Quando utilizar}
   \begin{itemize}
     \item Quando não se conhece a expressão analítica da função e sim apenas o seu valor em alguns pontos
     \item Quando a função é extremamente complexa e de difícil trabalho
   \end{itemize}
\end{frame}


\begin{frame}
  \frametitle{Polinômios de Lagrange}
  \begin{center}
  \begin{equation} 
      L_i(x) = \displaystyle\prod_{\substack{
        k = 0 \\
        k \neq i
        }}^n
        \frac{x - x_k}{x_i - x_k}, 0 \leq i \leq n
   \end{equation}
   \begin{equation}
    P_n(x) = \displaystyle\sum_{i=0}^{n}L_i(x)f(x_i)
   \end{equation}

   \end{center}
\end{frame}

\section{Trabalho}
\begin{frame}
  \frametitle{Função utilizada}
  \begin{center}
    $f(x)=sen(x), 0 \leq x \leq 10$
  \end{center}

\end{frame}

\begin{frame}[fragile]
  \frametitle{Dados base}
\begin{table}[]
  \begin{tabular}{|c|c|l|c|c|}
  \cline{1-2} \cline{4-5}
  f(x)                   & x &  & f(x)                          & x                       \\ \cline{1-2} \cline{4-5} 
  0.00000                & 0 &  & -0.95892                      & 5                       \\ \cline{1-2} \cline{4-5} 
  0.84147                & 1 &  & -0.27942                      & 6                       \\ \cline{1-2} \cline{4-5} 
  0.90930                & 2 &  & 0.65699                       & 7                       \\ \cline{1-2} \cline{4-5} 
  0.14112                & 3 &  & 0.98936                       & 8                       \\ \cline{1-2} \cline{4-5} 
  -0.75680               & 4 &  & 0.41212                       & 9                       \\ \cline{1-2} \cline{4-5} 
  \multicolumn{1}{|l|}{} &   &  & \multicolumn{1}{l|}{-0.54402} & \multicolumn{1}{l|}{10} \\ \cline{1-2} \cline{4-5} 
  \end{tabular}
  \end{table}

  \end{frame}

\begin{frame}[fragile]
    \frametitle{Resultados}
    \begin{center}
      \pgfuseimage{resultado}
  \end{center}
\end{frame}

\section{Referências}
\begin{frame}[fragile]
    \frametitle{Referências}
    \begin{footnotesize}
        \begin{verbatim}
          Numerical Analysis, 9 edition, Richard L. Burden, J. Douglas Faires
          Numerical Mathematics and Computing, 7 edition, Ward Cheney, David Kincaid
        \end{verbatim}
    \end{footnotesize}
\end{frame}
    
    
\section{Dúvidas}
\begin{frame}[fragile]
    \frametitle{Perguntas}
        \begin{center}
            \pgfuseimage{duvidas}
        \end{center}

\end{frame}
    
\end{document}
    
